
\documentclass[12pt]{article}  % <-- updated here
\pagestyle{plain}
\renewcommand{\baselinestretch}{1.3}
\topmargin -15 mm
\oddsidemargin 0mm
\textwidth 165mm
\textheight 230mm

\def\be{\begin{equation}}
\def\ee{\end{equation}}
\def\bea{\begin{eqnarray}}
\def\eea{\end{eqnarray}}

\usepackage{amsmath}
\usepackage{graphicx}
\usepackage{amsmath}
\usepackage{float}

\begin{document}

\begin{center}
{\Large{\bf A Unified Scale Factor for the Cosmic Evolution -Motivated by Brane World Models-}}

\vskip .5cm
{\large Farzin Safarzadeh-Maleki }
\vskip .1cm
{\it School of Astronomy, Institute for Research in Fundamental Sciences (IPM) \\
P.O.Box: 19395-5531, Tehran, Iran}\\
{\sl e-mails:  f.safarzadeh@aut.ac.ir}\\
\end{center}

\begin{abstract}

This paper introduces a novel cosmological scale factor, $a(t)=e^{H(t)} { (1-e^{-k(t)t}) }^{b(t)}$,  as a strong candidate for effectively modeling the universe in a unified and continuous description across all major epochs of cosmic evolution. We provide a detailed analysis, supported by graphical representations of the universe's evolution with smooth transition from the early inflation to the present acceleration expansion, proposing a coherent and comprehensive picture of cosmic history, consistent with current observational constraints. Furthermore, we explore the theoretical foundations of this model within extra-dimensional physics, demonstrating its compatibility with the brane-world paradigm. Through analytical approximations, the model describes a natural transition from a high-energy, brane-dominated early universe to the low-energy regime governed by general relativity, providing a compelling narrative of cosmic evolution across disparate energy scales. Additionally, the explicit presence of quantum statistical corrections to the Hubble parameter of this scale factor, as a prominent feature of the model, offers a phenomenological approach to incorporating quantum statistic into the classical evolution of the universe. 




\end{abstract}


\
\
\



{\it PACS numbers}: 98.80. ± k; 11.25. ±w

{\it Keywords}: Cosmology; Scale factor;
Brane Cosmology; Evolution of the universe.

\vskip .5cm



\section{Introduction}

A central goal in modern theoretical cosmology is to construct a coherent and unified description of the universe’s expansion history. Its significance lies not only in mathematical elegance, but in its capacity to synthesize the disparate phases of cosmic history into a coherent, predictive, and physically interpretable framework. In this regard, various methodologies have been proposed cosmological models trying to describe the universe's expansion history within a single framework. For example, quintessential inflation models aim to unify the early and late accelerated phases through a single scalar field \cite{1}-\cite{6}. Investigations in modified gravity theories also underscore the desire to capture the full cosmic history and describe the early-time inflation and late-time acceleration \cite{7}-\cite{9}. Additionally, some studies suggest parametrizing the Hubble rate or scale factor to achieve a unified framework \cite{10}-\cite{12}. Despite these extensive researches, a single continuous formula that can provide a unified description for the universe throughout its whole evolution, remains elusive.
\
\


On the other hand, scale factor, $a(t)$, as a key parameter for understanding the dynamics and observational signatures of the Friedmann–Lemaître–Robertson–Walker (FLRW) spacetime, describes how the size of the universe changes over time and allowing cosmologists to connect theoretical models with observational data and map cosmic evolution  \cite {13,14}. The functional form of the scale factor is dictated by the universe's composition and the dominant energy components driving expansion. It is typically modeled piecewise, with distinct form for each era - inflation ($a(t) \propto e^{H_{inf} t}$), radiation domination ($a(t) \propto t^{1/2}$), matter domination ($a(t) \propto  t^{2/3}$), and late-time accelerated expansion ($a(t) \propto e^{H_{ \Lambda} t}$)  \cite{15}-\cite{27}. While analytic solutions for $a(t)$ exist for individual epochs, no single connected expression encapsulates the entire cosmic evolution and the transition between them is often described using patched functions, leading to discontinuities and complicating theoretical modeling and numerical simulations. Therefore, a unified formula for the scale factor is not only a mathematical convenience, but it is a potential key to uncovering a more profound cosmological narrative, one in which the universe’s past, present, and future are aspects of a single and coherent dynamical law.

To address these challenges, we propose a novel, analytically tractable scale factor $ a(t)=e^{H(t)} { (1-e^{-k(t)t}) }^{b(t)}$, that interpolates between the four standard cosmological epochs and recovers the correct asymptotic behaviors in each era. Furthermore, in order to preserve the monotonic growth expected in the scale factor throughout cosmic evolution and to ensure continuous differentiability at transition points, we employ smooth sigmoid transitions \cite{28}-\cite{30} in reconstructing the scale factor’s key parameters. We illustrate the scale factor's evolution through cosmic history, with global and zoomed views, exhibiting strong agreement with supernova, cosmic microwave background (CMB), and baryon acoustic oscillation(BAO) observations \cite{31}-\cite{36}. 

A distinguished feature of this model is its novel modified Hubble parameter that incorporates quantum-statistical corrections via a Bose-Einstein distribution and a particle pressure term.This quantum-informed perspective offers a departure from conventional, purely geometric, classical models, promising a refined understanding of cosmological expansion.

Beyond its mathematical elegance, this model provides a physically motivated, brane-world description of cosmic evolution from inflation to late-time acceleration, revealing a deeper physical structure. Specifically, the model's key parameter, $k(t)$ times time, allows for a dual interpretation of cosmic evolution within brane cosmology, encompassing both Randall–Sundrum type-II (RS-II) scenarios \cite{37}-\cite{39} and variable brane tension models \cite{40}-\cite{43}. This framework smoothly transitions from a high-energy, brane-dominated early universe to low-energy general relativity, potentially offering insights into a unified theory.



This article proceeds as follows: Section 2 introduces the new scale factor. Then its cosmic evolution across various epochs will be discussed. Section 3 proposes a continuous version of the unified scale factor description, employing proper sigmoid functions to smoothly interpolate between different epochs, analyzes its cosmological implications and provides plots demonstrating observational agreement.  Section 4 explores the model's physical motivation within the brane cosmology framework. Finally Section 5 is devoted to the conclusions.



%%%%%%%%%%%%%%%%%%%%%%%%%%%%%%%%%%%%%%%%%%%%%%%%%%%%%%%%%%%%%%%%%%%%%
\section{	Introducing the Scale Factor }



In order to unify the exponential expansion of inflation and dark energy domination with intermediate radiation and matter-dominated deceleration phases, we propose a new scale factor as
\bea
 a(t)=e^{H(t)} { (1-e^{-k(t)t}) }^{b(t)},
\eea

The first term, $e^{H(t)}$, mimics the scale factor of inflationary models, with dynamically evolving parameter of $H(t)$ which models the universe’s expansion directly as:

\bea
	H(t) =
	\begin{cases}
		H_{\mathrm{inf}} =  10^{37} \, \mathrm{s}^{-1}, & 10^{-36} < t \leq 10^{-32} \\[8pt]
		H_{\mathrm{inter}} = \dfrac{b(t)}{t} \, \mathrm{s}^{-1}, & 10^{-32} < t < 10^{17} \\[8pt]
		H_{\mathrm{DE}} = 2.2 \times 10^{-18} \, \mathrm{s}^{-1}, & t \geq 10^{17}
	\end{cases}
\eea

This parameter controls the exponential expansion and encodes dynamical scaling behavior that resembles Hubble-like evolution. The subindices "inf," "inter," and "DE" denote inflation, intermediate stages, and dark energy era, respectively and corresponding values are based on recent Planck and BAO data \cite {34}. In addition, 
\bea
	b(t) = \frac{1}{2} \frac{1}{1 + \frac{t}{t_{\mathrm{eq}}}} + \frac{2}{3} \frac{\frac{t}{t_{\mathrm{eq}}}}{1 + \frac{t}{t_{\mathrm{eq}}}},
\eea
allows the scale factor to follow the correct power law evolution in each period while ensuring a smooth connection. In this relation, $t_{\mathrm{eq}} \approx 10^{12}$s, is the radiation-matter equality time. 


The second term, the power-law like modulator $ { (1-e^{-k(t)t}) }^{b(t)}$, dominates at intermediate times, enabling transitions between radiation and matter eras while matching their standard expansion. In this relation, $k(t)$ is defined as
\bea
	k(t) =
	\begin{cases}
		k_{\mathrm{inf}} \geq 10^{39}, & 10^{-36} < t \leq 10^{-32} \\[8pt]
		k_{\mathrm{inter}} = \frac{1}{e\, b(t)} \sqrt{\frac{8 \pi G}{3} \rho_0^{\mathrm{RD,MD}}}, & 10^{-32} < t < 10^{17} \\[8pt]
		k_{\mathrm{DE}} \geq 10^{-14}, & t \geq 10^{17}
	\end{cases}
\eea

and serves as a scale modulating parameter within the energy evolution function and controls the rate at which the exponential contribution saturates. In Eq.~(4), the inflation and dark energy values are not arbitrary. The model's physics dictates these lower bounds to ensure $kt\gg1$ and the desired exponential expansion. This will be further discussed in subsequent cosmological analysis. In addition, during the intermediate epoch, $G=6.67430  \times 10^{-11} \frac{N m^2}{kg^2}$ is the 2018 CODATA Newtonian constant of gravitation and  $\rho_{0}^{(RD,MD)}$ represents today's energy density value of radiation or matter. Specifically, during radiation-dominated (RD) era ($10^{-32} \leq t \leq 10^{12}$) we have $\rho_{0}^{RD}\approx 7.86 \times 10^{-31} \frac{kg}{m^{3}}$, while during matter-dominated (MD) epoch ($10^{12} \leq t \leq 10^{17}$), $ \rho_{0}^{MD}\approx 2.68 \times 10^{-27}  \frac{kg}{m^{3}}$  \cite{34}.

The following section discusses the proposed scale factor's behavior in different cosmological eras.

\subsection{ Cosmological Analysis }
	

In the very early universe where $(kt)_{inf}\gg1$, the damping factor $e^{-kt}$ approaches zero leading to a scale factor approximates as $ a_{inf} (t)  \approx e^{(H_{inf} t)}$. This represents the initial exponential expansion of the inflation. 
At intermediate phases where $(kt)_{inter}\ll1, e^{-k(t)t}\approx 1-k(t)t$, then according to each era, the corresponding scale factor can be generated as follows: 
During radiation-dominated era, $t \ll t_{\mathrm{eq}}$ implies $1 + \frac{t}{t_{\mathrm{eq}}} \approx 1 $  in Eq.~(3), thus, $b_{\mathrm{RD}}(t) = \frac{1}{2}$. From Eqs.~(1)--(4), the corresponding scale factor can be produced as $a_{\mathrm{RD}}(t) = \left( 2 \sqrt{\frac{8 \pi G}{3} \rho_0^{\mathrm{RD}}} \, t \right)^{\frac{1}{2}}$,
consistent with the standard Friedmann equation prediction for a radiation-dominated universe $\left(a_{\mathrm{RD}} \propto t^{1/2}\right)$. 
Similarly, in the matter-dominated era, $ t \gg t_{\mathrm{eq}} $ in Eq.~(3) implies $ 1 + \frac{t}{t_{\mathrm{eq}}} \approx \frac{t}{t_{\mathrm{eq}}} $, thus, $b_{\mathrm{MD
}}(t) = \frac{2}{3}$.
The corresponding scale factor is then $a_{\mathrm{MD}}(t) \approx \left( \frac{3}{2} \sqrt{\frac{8 \pi G}{3} \rho_0^{\mathrm{MD}}} \, t \right)^{\frac{2}{3}}$,
which aligns with the Friedmann equation prediction for a matter-dominated universe $\left(a_{\mathrm{MD}} \propto t^{2/3}\right)$. Therefore, the intermediate behavior of the scale factor can be unified into a single formula 
\bea
a_{\mathrm{RD,MD}}(t) = \left( \frac{\sqrt{\frac{8 \pi G}{3} \rho_0^{\mathrm{RD,MD}}} \, t}{b(t)} \right)^{b(t)}.
\eea
At late times, ($t\rightarrow \infty$), the damping term vanishes exponentially fast, leaving $e^{H_{DE} t}$ to dominate the late time evolution. This asymptotic behavior naturally fits observational evidence of a universe dominated by dark energy.

 
While the model accurately reproduces the major cosmological epochs – inflation, radiation/matter and dark energy domination, aligning with observations \cite{31}-\cite{36} – it exhibits discontinuities in the scale factor and its key parameters between these epochs. To ensure physical plausibility and numerical stability, these parameters must be reconstructed to be continuous and differentiable across epoch transitions, preserving the monotonic growth of the scale factor. This smoothing is particularly crucial for early-universe inflation models, where even small discontinuities can be significantly amplified. The next section discusses how to achieve this smooth and continuous evolution by eliminating patches and jumps.


\section{Exploring a Connected Smooth Unified Scale Factor}

To achieve a continuous and smooth scale factor description across all cosmic epochs, we employ the sigmoid function $\sigma(y) = \frac{1}{1 + e^{y}}$, ranging from 0 to 1. This function approximates the piecewise behavior of non-smooth dynamical systems, enabling the continuous and differentiable representation of piecewise functions \cite{28}-\cite{30}. 

In this model, we employ the sigmoid function $\sigma(t) = \frac{1}{1 + \exp\left(-\frac{t - t_{T}}{\Delta_{T}}\right)}$ to smoothly interpolate between epochs. Here, $t$ is cosmic time, \(t_T\) is the transition time and \(\Delta_T\) is the time scale free parameter which should be set such that the transition function can ensure a smooth behavior. 
In what follows we use such a sigmoid function to model the rapid, continuous transition of the scale factor's exponential growth rate during the inflationary epoch. In this function we defined $t_T$ as the time of inflationary phase (typically around $10^{-32}$ seconds), and $\Delta_T = 10^x$ that controls the steepness of transition. To ensure the transition occurs effectively within the inflationary window, we require the function to evolve from $\sigma(t) \approx 0.01$ to $\sigma(t) \approx 0.99$ across a physical duration $\Delta t$. This range corresponds to an interval in the sigmoid’s argument $y = (t - t_0)/10^x$, where the change $\Delta y \approx 9.2$ accounts for the rapid rise from 0.01 to 0.99. Therefore, we relate the time scale of the transition to steepness parameter via $\Delta t = 9.2 \cdot 10^x$.
Imposing a physically motivated duration for inflation consistent with $N=\int{H dt}\approx H.\Delta (t) \sim 60$ e-foldings and a Hubble parameter $H \sim 10^{37} \, \text{s}^{-1}$, we infer $\Delta t \approx 6 \times 10^{-36} \, \text{s}$, which yields $10^x \approx \frac{6 \times 10^{-36}}{9.2} \Rightarrow x \approx -36.186$. This parameterization captures both the rapid nature of the inflationary transition and aligns with standard cosmological expectations for the number of e-foldings generated during the epoch. It provides a compact, analytically tractable form suitable for embedding within effective field models or numerical simulations of early-universe dynamics. 

Accordingly, epoch transitions are defined by the following functions:

Inflation to radiation:
\bea
\sigma_{\mathrm{inf-RD}}(t) = \frac{1}{1 + e^{-\frac{t - t_{\mathrm{inf}}}{\Delta_{\mathrm{inf}}}}}, \quad \text{with} \quad t_{\mathrm{inf}} \sim 10^{-32}, \quad \Delta_{\mathrm{inf}} \sim 10^{-36}
\eea

Radiation to matter:
\bea
\sigma_{\mathrm{RD-MD}}(t) = \frac{1}{1 + e^{-\frac{t - t_{\mathrm{eq}}}{\Delta_{\mathrm{eq}}}}}, \quad \text{with} \quad t_{\mathrm{eq}} \sim 10^{12}, \quad \Delta_{\mathrm{eq}} \sim 10^{11}
\eea
Matter to dark energy:
\bea
\sigma_{\mathrm{MD-DE}}(t) = \frac{1}{1 + e^{-\frac{t - t_{\mathrm{DE}}}{\Delta_{\mathrm{DE}}}}}, \quad \text{with} \quad t_{\mathrm{DE}} \sim 10^{17}, \quad \Delta_{\mathrm{DE}} \sim 10^{16}
\eea
These functions accurately capture the transitions between epochs at the desired times. Specifically, during inflation, e.g., $t=10^{-34}\,\mathrm{s}$, we find $\sigma_{\mathrm{inf-RD}}(t) \approx 0$, $\sigma_{\mathrm{RD-MD}}(t) \approx 0$, and \(\sigma_{\mathrm{MD-DE}}(t) \approx 0\). During radiation domination, e.g., \(t=10^{-20}\,\mathrm{s}\), we have \(\sigma_{\mathrm{inf-RD}}(t) \approx 1\), \(\sigma_{\mathrm{RD-MD}}(t) \approx 0\), and \(\sigma_{\mathrm{MD-DE}}(t) \approx 0\). Similarly, for matter domination, \(\sigma_{\mathrm{inf-RD}}(t) \approx 1\), \(\sigma_{\mathrm{RD-MD}}(t) \approx 1\), and \(\sigma_{\mathrm{MD-DE}}(t) \approx 0\), and at late times, \(\sigma_{\mathrm{inf-RD}}(t) \approx 1\), \(\sigma_{\mathrm{RD-MD}}(t) \approx 1\), and \(\sigma_{\mathrm{MD-DE}}(t) \approx 1\). These functions then allow the definition of connected functions for the scale factor's piecewise parameters, \(H(t)\) and \(k(t)\), as follows: 
\begin{align}
	H(t) &= H_{\mathrm{inf}} \bigl[1 - \sigma_{\mathrm{inf-RD}}(t)\bigr] 
	+ \sigma_{\mathrm{inf-RD}}(t) \biggl\{ 
	\bigl[1 - \sigma_{\mathrm{RD-MD}}(t)\bigr] H_{\mathrm{RD}}(t) \notag \\
	&\quad + \sigma_{\mathrm{RD-MD}}(t) \Big[ \bigl(1 - \sigma_{\mathrm{DE}}(t)\bigr) H_{\mathrm{MD}}(t) 	+ H_{\mathrm{DE}}(t) \sigma_{\mathrm{DE}}(t) \Big]
	\biggr\},
	\label{eq:H}
\end{align}
which serves as a structural interpolation function controlling the early time inflationary phase and late time acceleration through exponential behavior, and
 \begin{align}
	k(t) &= k_{\mathrm{inf}} \bigl[1 - \sigma_{\mathrm{inf-RD}}(t)\bigr] 
	+ k_{\mathrm{RD}}\, \sigma_{\mathrm{inf-RD}}(t) \bigl[1 - \sigma_{\mathrm{RD-MD}}(t)\bigr] \notag \\
	&\quad + k_{\mathrm{MD}}\, \sigma_{\mathrm{RD-MD}}(t) \bigl[1 - \sigma_{\mathrm{MD-DE}}(t)\bigr] 
	+ k_{\mathrm{DM}}\, \sigma_{\mathrm{MD-DE}}(t).
	\label{eq:k}
\end{align}
as a key parameter governing the energy injection or dissipation, depending on the era.

The following plots show evolution of the modified parameters as a connected smooth functions: 
Figure 1 (upper left) illustrates the evolving Hubble parameter, \(H(t)\), over cosmic time. Starting at a high value during inflation, \(H(t)\) decreases as \(\frac{b(t)}{t}\) throughout the radiation and matter eras, eventually stabilizing to a small constant value during the dark energy era. This sigmoidal decrease from \(H_{\mathrm{Inf}} \to H_{\mathrm{DE}}\) reflects the decay of effective vacuum energy density from inflationary to the present dark energy scale. 
Figure 1 (upper right) illustrates the evolution of $k(t)$ over cosmic time, a parameter that activates the power-law phase. Its curve sharply suppresses the power-law at early times, rapidly drops at the end of inflation to initiate power-law dominance, remains low during radiation and matter eras, and then increases at late times to suppress the power-law contribution, reverting to exponential expansion. We will see later that this parameter plays a key role in the physical interpretation of the model in brane world scenarios.


\begin{figure}[H]
	\centering
	\includegraphics[width=0.45\textwidth]{H_plot.jpg}
	\hfill
	\includegraphics[width=0.45\textwidth]{k_plot.jpg}
	
	\vskip\baselineskip
	
	\includegraphics[width=0.45\textwidth]{b_plot.jpg}
	
	\caption{ Evolution of the key parameters of the model during cosmic eras }
	\label{fig:four_figures}
\end{figure}

\


It should be noted that $b(t)$ is already continues because it smoothly transitions between 1/2 and 2/3 (Figure 1 (bottom)). However, the factor $(1-e^{-k(t)t})^{b(t)}$ can cause the scale factor to be non-monotonic or numerically unstable especially at very small or large $t$. For example, if $k(t)$ is small, then $1-e^{-k(t)t}\approx k(t)t$, and raising it to a power could introduce sharp changes. This issue can be solved by multiplying this factor to the numerically integrated base scale factor $a_{num}(t)$. Therefore, the scale factor defines as $a(t)=a_{num}(t)\times [1-e^{-k(t)t}]^{b(t)}$, where $a_{num}(t)$ is from integrating $\frac{da}{dt}=a H(t)$ with initial condition adjusted to $a_{num}(t_0)=\frac{1}{[1-e^{-k(t_0)t_0}]^{b(t_0)}}$, which normalizes $a(t_0)$ to 1.


These formulations accurately reproduce expected values and behaviors across all eras, ensuring continuous and smooth transitions and naturally interpolating between them without piecewise functions. To properly capture the evolution of scale factor across different cosmological epochs, we present both a global view (Figures 2,4) and zoomed-in plots (Figure 3). The global log-log plot clearly demonstrates the dramatic exponential growth during inflation, while post-inflationary epochs appear flatter due to the relatively slower power-law expansion. This flattening is a visual artifact of the wide time span rather than a physical suppression of growth. In such a wide range, even the rapid growths after inflation appear smooth. 

\
\
\
\



\begin{figure}[H]
	\centering
	\includegraphics[width=0.7\linewidth]{a_total.jpg}
	\caption{Evolution of the scale factor as a function of cosmic time, spanning from inflation ($t\approx10^{-35}$ s) to the present-day dark energy dominated era ($t\approx10^{18}$ s). Both axes are logarithmic to accommodate the vast temporal range. Although the scale factor continues to increase during radiation, matter, and dark energy-dominated epochs, the visual slope flattens due to the slower growth rates compared to the inflationary exponential expansion.}
	\label{fig:a_total}
\end{figure}



\begin{figure}[H]
	\centering
	\includegraphics[width=0.45\textwidth]{inflation.jpg}
	\hfill
	\includegraphics[width=0.45\textwidth]{radiation.jpg}
	
	\vskip\baselineskip
	
	\includegraphics[width=0.45\textwidth]{matter.jpg}
	\hfill
	\includegraphics[width=0.45\textwidth]{dark.jpg}
	
	\caption{ Evolution of the scale factor in each epoch separately }
	\label{fig:four_figures}
\end{figure}


\
\
\


\begin{figure}[H]
	\centering
	\includegraphics[width=0.7\linewidth]{a_total_colored.jpg}
	\caption{integrating all above eras into a unified plot}
	\label{fig:final}
\end{figure}



Therefore, this modified scale factor offers a unified, continuous, global formulation across major cosmological epochs, replacing traditional piecewise approaches. It can address key cosmological challenges, including the horizon, flatness, and monopole problems, by incorporating a period of exponential inflation. It then accurately describes the intermediate expansion history and finally, reproduces the effects of dark energy through late-time exponential expansion, demonstrating strong agreement with observational data from supernova, CMB, and BAO observations \cite{31}-\cite{36}.

In the following section we demonstrate the model’s potential to provide deeper physical insight and a unified description of the universe's evolution within brane world cosmology, going beyond a purely mathematical approach.



\section{Motivation and Theoretical Background}  

The present model allows interpretation within 5D brane-world scenarios. In order to show their common features, let us derive the effective Hubble parameter of the model, denoted by $\mathcal H(t)$, using the scale factor definition in Eq.~(1):  
\bea 
{\mathcal H(t)} = \frac{\dot{a}(t)}{a(t)} = H(t) + \dot{H}(t) t + \dot{b}(t) \ln\left(1 - e^{-k(t)t}\right) + b(t) \frac{e^{-k(t) t}}{1 - e^{-k(t) t}} \left[k(t) + t \dot{k}(t)\right]  
\eea

The time-dependent Hubble parameter presented here, potentially arising from brane-world effects, suggests modifications to the standard cosmological expansion rate and is one of the most interesting results found in this work.

In this relation, the term $H(t) + \dot{H}(t)t$ reflects not only the expansion rate but also its evolution, and thus allows for a more accurate description of the universe expansion. Integrating this term into the Hubble parameter formalism allows for sensitive detection of deviations from standard expansion histories.
The subsequent terms, by assuming $k(t)t\propto\frac{\varepsilon}{T}$, incorporate the mean occupation number for Bose-Einstein statistics at full quantum area, $<n_\varepsilon>=\frac{1}{e^{\frac{\varepsilon}{T}}-1}$, and particle pressure $\frac{PV}{T}=-\sum \ln(1-e^{\frac{-\varepsilon}{T}})$ \cite{37}, explicitly as novel corrections to the modified Hubble parameter. This implies a direct influence of quantum statistics on the cosmological expansion as a distinguishing feature of the model from traditional, geometrically motivated classical corrections. Specifically, the function $\ln(1 - e^{-k(t)t})$ links to entropy production mechanisms arising from the cosmic plasma's thermal fluctuations. These fluctuations manifest as logarithmic corrections to the effective cosmological parameters and entropy content, enriching the phenomenology beyond classical Friedmann-Robertson-Walker (FRW) evolution. Within the framework of FRW brane cosmology, subdominant logarithmic corrections, arising from thermal considerations, induce modifications to the entropy-area relation, thereby impacting the cosmological dynamics \cite{38},\cite{39}. In the present model, this term exhibits distinct asymptotic behaviors across cosmological epochs. During inflation, the logarithmic term is negligible due to the smallness of $e^{k(t)t}$, having minimal impact on background dynamics. However, at intermediate stages, when $k(t)t$ is moderate, the term approximates to $\ln(k(t)t)$, thereby applying significant logarithmic corrections.
In the next term, the factor $\frac{e^{-k(t)t}}{1 - e^{-k(t)t}}=\frac{1}{e^{k(t)t}-1}$ resembles  Bose-Einstein distribution, suggesting potential contributions from particle production or thermal effects that may directly influence cosmological expansion. A consequence of this quantum distribution is the emergence of a Bose-Einstein condensate, where the ground state becomes  occupied at sufficiently low temperatures or high particle densities. Some studies have suggested that quantum-statistical effects, especially those originating from Bose--Einstein condensates may have left imprints on cosmic expansion. It is shown that a Bose-Einstein condensation of ultralight bosons can account for the dark matter content of our universe, while its associated quantum potential can account for dark energy \cite{40}. 

Recent studies demonstrate that, two parametrizations of the Hubble parameter, the power-law and logarithmic corrections, provide a convincing fit to current observational data. These corrections effectively model observed cosmic acceleration and refine the standard 
$\Lambda$CDM model \cite{41}. Our modified Hubble parameter inherently produces such corrections.

	

\subsection{Correspondence with Brane-World Cosmology} 
In brane-world cosmologies, our observable universe is realized as a 3-brane embedded in a higher-dimensional bulk. The effective four-dimensional dynamics on this brane are governed by modified Friedmann equations that include corrections due to the brane tension and extra-dimensional effects. In the Randall-Sundrum II-type brane-world models, the Friedmann equation on the 3-brane is modified as \cite{42}-\cite{44}:  
\bea  
	H_{\mathrm{Phys}}^{2} = \frac{8 \pi G_4}{3} \rho \left(1 + \frac{\rho}{2 \lambda}\right) + \frac{\Lambda_4}{3} + \frac{\varepsilon}{a^4},   
\eea 
where $H_{\mathrm{Phys}}$ is the Hubble parameter on the brane, $\rho$ is the energy density on the brane, $G_4$ is the effective 4D Newton’s constant, $\lambda$ is the brane tension, $\Lambda_4$ is an effective 4D cosmological constant, and $\varepsilon / a^{4}$ is the dark radiation term. At high energies $(\rho \gg \lambda)$, the $\rho^2 / \lambda$ term dominates, leading to  $	H_{\mathrm{Phys}}^{2} \approx \frac{4 \pi G}{3 \lambda} \rho^{2}$, modifying expansion behavior significantly, while at low energies $(\rho \ll \lambda)$, the standard Friedmann behavior is recovered: $	H_{\mathrm{Phys}}^2 \approx \frac{8 \pi G}{3} \rho $.  



In what follows, we will discuss the cosmological implications of this model in accordance with the brane world model in each period separately:

During the early universe (at high energies), where $k(t) t\gg 1$, the exponential term $e^{-k(t) t}$ vanishes, leading to  $	{\mathcal H}^2(t) \approx \left( H(t) + \dot{H}(t) t  \right)^2 $.
From Eq.~(2), $H(t) \approx H_{\mathrm{inf}} = \mathrm{const}$, then, $	{\mathcal H}^2(t) \approx H_{\mathrm{inf}}^2$.  
This quadratic structure, resembling high energy corrections in RS-II brane-world models, is captured by the exponential term in our ansatz. Consequently, the brane inflationary density becomes $\rho_{\mathrm{eff}} = \sqrt{\frac{3 \lambda}{4 \pi G}} H_{\mathrm{inf}}$.  
During the universe's intermediate expansion era $\bigl(k(t) t \ll 1\bigr)$, where $1 - e^{-k(t) t} \approx k(t) t$, the effective Hubble parameter can be approximated as  $ {\mathcal H(t)} \approx H(t) + \dot{H}(t) t + \dot{b}(t) \ln(k(t)t) +\frac{b(t)}{t}\left[1+t\eta(t) - k(t)t-k(t)t^2 \eta(t) \right]$, 
with $\eta(t)=\frac{ \dot{k}(t)}{k(t)}$. Since other terms are subdominant compared to $\frac{b(t)}{t}$ during power-law expansion, then, ${\mathcal H}(t)\sim \frac{b(t)}{t}$. This means, the standard linear term in the Friedmann equation dominates, leading to radiation and matter-dominated epochs characterized by the energy density dependence $\rho_{eff}(t)=\frac{3{\mathcal H}^{2}(t)}{8\pi G}\approx \frac{3b^{2}(t)}{8\pi G t^2}$. This is consistent with RS-II model’s low-energy limit, where the brane corrections become insignificant and the cosmology effectively transitions to standard 4D general relativity behavior.
At late times, this model exhibits accelerated expansion again, as the Hubble parameter approaches a constant value, $\mathcal H(t)\approx H_{DE}$ with increasing $k$. The corresponding effective energy density $\rho_{eff}(t)=\dfrac{3H_{DE}^{2}}{8\pi G}$, asymptotically approaches a constant. This residual cosmological constant can be described as a small energy density on the brane, similar to dark energy.

These asymptotic behaviors demonstrate that the universe's evolution is naturally driven by the model's key parameter, $k(t)t$. Physical interpretation of this component is discussed further in the next section.



\subsection{Physical Interpretation of $kt$ in Brane Cosmology}

By defining  $kt \propto \frac{\rho(t)}{\lambda(t)}$, one can derive the relative strength of brane corrections that governs the onset and suppression of brane‐world cosmological corrections across all epochs \cite{45}-\cite{46}. To understand this physical concept within our model, first, let's apply it to the last term of the Hubble parameter Eq.(11): 
Assuming the map $\frac{\varepsilon}{T}\mapsto \frac{\rho(t)}{\lambda}$, prefactor $\frac{1}{e^{\frac{\rho(t)}{\lambda}} – 1}$ can be interpreted as a quantum-statistical weight factor affecting the energy contribution of bosonic matter to cosmological dynamics.
Its behavior across cosmic epochs can be summarized as follows: During inflation ($\rho(t)/\lambda \gg 1$), the Bose-Einstein factor is exponentially suppressed,	$\frac{1}{e^{\rho(t)/\lambda} - 1} \approx 0$, resulting in negligible quantum bosonic occupation and dominance of classical dynamics. 
At reheating and transition epochs ($\rho(t)/\lambda \sim \mathcal{O}(1)$), The quantum bosonic distribution function gradually begins to grow. 
In radiation and matter dominated eras ($\rho(t)/\lambda \ll 1$), the Bose-Einstein factor approximates to $\frac{1}{e^{\rho(t)/\lambda} - 1} \approx \frac{\lambda}{\rho(t)}$, indicating a large bosonic occupation and high condensation probability. This aligns with \cite {47}, which shows that a Bose-Einstein Condensation (BEC) phase transition can occur around redshift $z_{\text{BEC}} \approx 1200$, suggesting that quantum effects become dynamically relevant during or after the recombination epoch. 
Finally, in the dark energy dominated era ($\rho(t) \to \text{const.}$, $\rho/\lambda \ll 1$), the correction term asymptotes to a large static value, but its physical impact is reduced by the dominant cosmological constant. 
Therefore, existence of a Bose-Einstein-like correction into the Hubble parameter of this model, offers a novel phenomenological approach to brane-world cosmology, by effectively modeling the quantum-statistical response of the universe's expansion rate to density variations.


Next factor $\left( \frac{\rho}{\lambda} + t \frac{d}{dt} \left( \frac{\rho}{\lambda} \right) \right)$ incorporates both the magnitude and rate of change of energy density with respect to the brane tension, making the correction sensitive to dynamical transitions. The $\frac{d}{dt} (\frac{\rho}{\lambda})$ term, represents the time evolution of the energy density relative to the brane tension. Such time-dependent corrections can arise in scenarios where the brane exchanges energy with the bulk or where the brane tension itself evolves over time \cite{48}-\cite{51}.
In continue, we explore the cosmological interpretation of $kt$ over cosmic epochs, first, in constant and then in variable-tension brane-world models:

Scenarios with Constant Brane Tension (RS-Models): 
During the inflationary period, the energy density $\rho(t)$ was exceedingly high. Assuming a constant brane tension $\lambda$, this leads to $kt \gg 1$. In this regime, the quadratic corrections to the Friedmann equations become significant, enhancing the Hubble expansion rate and facilitating a rapid inflationary phase. This behavior is consistent with the modified Friedmann equation in RS brane-world scenarios. 
During the radiation-dominated era, decreasing energy density causes a decline in $kt$ (for constant $\lambda$). Big Bang nucleosynthesis (BBN) and CMB measurements require these corrections be minimal \cite{52}. This trend continues into the matter era as $\rho(t)$ further decreases, resulting in a smaller $kt$. This suppression ensures structure formation and CMB anisotropies evolve as in standard $\Lambda$CDM cosmology, consistent with Planck data \cite{53}. Consequently, with $kt\ll1$, quadratic corrections to the Friedmann equation become negligible, and the standard cosmological model is recovered.
At late times, the universe experiences accelerated expansion. In standard brane-world models with constant $\lambda$, $kt$ remains small due to the continued decrease of $\rho(t)$, ensuring consistency with $\Lambda$CDM predictions. 
These models require an additional dark energy component, such as a cosmological constant, to explain the observed late-time acceleration of the universe.

Scenarios with Varying Brane Tension: At the very early universe, this parameter is represented as $k_{inf}t\equiv \frac{\rho_{inf}}{\lambda}=H_{inf}\sqrt{\frac{3}{4\pi G \lambda}} $, where small tension of the brane (below inflationary energy density) leads to large $k_{inf}t$. This implies a soft brane, leading to significant high-energy corrections in the Friedmann equations. This regime can drive an inflationary expansion, as the quadratic energy density terms dominate the dynamics.
During the intermediate eras, when the universe expands and cools, the brane tension increases, reducing the influence of high-energy corrections, then, the dynamics are well-approximated by the standard Friedmann equation. This happens in high brane tension models ($\rho\ll \lambda$), effectively minimizing extra-dimensional effects, transitions to a standard 4D behavior and reproducing standard radiation and matter epochs. At late time, where brane tension evolves over time, $\lambda(t)$ decreases to small values, making large $kt$ once more, then, $H^{2}(t)\approx H_{DE}^{2}$. This leads to a dynamical dark energy component, driving the observed accelerated expansion of the universe without invoking a separate cosmological constant \cite{54}. It was shown in \cite{48} that a variable brane tension can explain dark energy where the brane tension was proposed to depend upon the scale factor. Our model expresses such a dependence as 
 $\lambda(t)=\frac{-\rho}{\ln[1-(\frac{a(t)}{e^{H(t)}})^{\frac{1}{b(t)}}]}.$

\section{Conclusion}	
This paper proposes a unified formula for the scale factor, $a(t)=e^{H(t)} (1-e^{-k(t)t})^{b(t)}$, a continuous and potentially analytic expression that smoothly interpolates across major cosmic epochs. By accurately capturing asymptotic behavior in each era, this formula offers a mathematically elegant alternative to piecewise modeling, providing conceptual cohesion and a deeper understanding of cosmic dynamics from the early universe to its ultimate fate, consistent with observational data from supernova, CMB, and BAO observations. Its simplicity and generality make it a versatile tool for analytical and numerical exploration of cosmological phenomena, enabling direct use in cosmic evolution simulations without artificial epoch patching.
Furthermore, this model is well-motivated within brane-world cosmology. It can provide a unified framework for rapid inflation, standard cosmological eras, and late-time accelerated expansion attributable to dark energy within both constant and variable-tension brane cosmological models. In the former, cosmic evolution arises from the natural decrease in energy density, while in the latter, it's achieved by choosing a suitable time-dependent brane tension. This dual interpretation is indeed due to the non-monotonic evolution of the key parameter of the model, $k(t)$ times time, inversely proportional to brane tension, and offers opportunities for further theoretical development. 
In addition, the modified Hubble parameter, derived from this scale factor, introduces quantum statistical corrections to the Hubble expansion. By explicitly including Bose-Einstein distribution and particle pressure, these corrections offer a phenomenological approach to embedding quantum statistic into the classical evolution of the universe. This yields refined expansion histories and structure formation scenarios that may resolve persistent challenges in $\Lambda$CDM cosmology, linking quantum microphysics and cosmic evolution.
In summary, this unified scale factor simplifies the mathematical and physical treatment of cosmic history, potentially revealing new insights into cosmological evolution's underlying structure.


{\bf Acknowledgement}:

The author gratefully acknowledges H. Firouzjahi, A. Talebian, and M. H. Namjoo for their valuable discussions that contributed to this work. 


\begin{thebibliography}{99}

\bibitem{1}
 Dimopoulos, Konstantinos, and J. W. F. Valle. "Modeling quintessential inflation." Astroparticle Physics 18, no. 3 (2002): 287-306 [astro-ph/0111417] and references therein.
\bibitem{2}
 M. Sami and V. Sahni, “Quintessential inflation on the brane and the relic gravity wave background,” Phys. Rev. D 70, 083513 (2004) [hep-th/0402086]. 
\bibitem{3}
 de Haro, Jaume, Jaume Amorós, and Supriya Pan. "Simple inflationary quintessential model." Physical Review D 93, no. 8 (2016): 084018. 
\bibitem{4}
 P. J. E. Peebles and A. Vilenkin, “Noninteracting dark matter,” Phys. Rev. D 60,103506 (1999) [astro-ph/9904396]. 
\bibitem{5}
 E. J. Copeland, A. R. Liddle and J. E. Lidsey, “Steep inflation: Ending brane world inflation by gravitational particle production,” Phys. Rev. D 64, 023509 (2001) [as tro-ph/0006421].
\bibitem{6}
 Hossain, Md Wali, R. Myrzakulov, M. Sami, and Emmanuel N. Saridakis. "Unification of inflation and dark energy à la quintessential inflation." International Journal of Modern Physics D 24, no. 05 (2015): 1530014 [gr-qc/1410.6100]. 
\bibitem{7}
 Nojiri, Shin’ichi, and Sergei D. Odintsov. "Modified gravity with negative and positive powers of curvature: Unification of inflation and cosmic acceleration." physical Review D 68, no. 12 (2003) 123512 [hep-th/0307288]
\bibitem{8}
 Nojiri, Shin’ichi, and Sergei D. Odintsov. "Unified cosmic history in modified gravity: from F (R) theory to Lorentz non-invariant models." Physics Reports 505, no. 2-4 (2011): 59-144.
\bibitem{9} Nojiri, S. S. D. O., S. D. Odintsov, and VK3683913 Oikonomou. "Modified gravity theories on a nutshell: Inflation, bounce and late-time evolution." Physics Reports 692 (2017): 1-104. And references in it.
\bibitem{10} Aydiner, E., I. Basaran-Öz, T. Dereli, and M. Sarisaman. "Late time transition of Universe and the hybrid scale factor." The European Physical Journal C 82, no. 1 (2022): 39.
\bibitem{11}
 Singh, Ashutosh, and R. Chaubey. "Unified and bouncing cosmologies with inhomogeneous viscous fluid." Astrophysics and Space Science 366, no. 1 (2021): 15.
\bibitem{12}
 Sazhin, M. V., O. S. Sazhina, and U. Chadayammuri. "The Scale Factor in the Universe with Dark Energy. arXiv: astro-ph." CO 1109 (2011): v1.
\bibitem{13}
 S. Weinberg. Cosmology. OUP Oxford, 2008.
\bibitem{14}
 D. Baumann, "TASI lectures on inflation" (2009).
\bibitem{15}
 A. Guth, Phys. Rev. D 23, 347 (1981); A. Linde, Phys. Lett. B 108, 389 (1982). 
\bibitem{16}
 P. A. R. Ade et al. [Planck Collaboration], Astron. Astrophys. 571, A22 (2014) [arXiv:1303.5082]
\bibitem{17}
 R. H. Brandenberger, “Cosmology of the Very Early Universe,” AIP Conf. Proc. 1268, 3-70 (2010) [arXiv:1003.1745 [hep-th]].
\bibitem{18}
Guth AH, “The Inflationary Universe: A Possible Solution To The Horizon And Flatness Problems”, Phys. Rev. D 23, 347 (1981).
\bibitem{19}
 R. Brout, F. Englert and E. Gunzig, “The Creation Of The Universe As A Quantum Phenomenon”, Annals Phys. 115, 78 (1978).
\bibitem{20}
 A. A. Starobinsky, “A New Type Of Isotropic Cosmological Models Without Singularity,” Phys. Lett.B 91, 99 (1980).
\bibitem{21}
 K. Sato, “First Order Phase Transition Of A Vacuum And Expansion Of The Universe,” Mon. Not.Roy. Astron. Soc. 195, 467 (1981).
\bibitem{22}
 Mukhanov, V. Physical Foundations of Cosmology, Cambridge University Press (2005); Mukhanov and G. Chibisov, “Quantum Fluctuation And Nonsingular Universe. (In Russian)”, JETP Lett. 33, 532 (1981) [Pisma Zh. Eksp. Teor. Fiz. 33, 549 (1981)].
\bibitem{23}
 A. D. Linde, Phys. Lett. B 108, 389 (1982); A. D. Linde, Phys. Rev. D 49, 748 (1994) [astro-ph/9307002].
\bibitem{24}
T. Padmanabhan, "Cosmological constant—the weight of the vacuum", Phys. Repts. 380 (2003)235. 
\bibitem{25}
 V. Sahni, "5 dark matter and dark energy." The Physics of the Early Universe (2004): 141-179.
\bibitem{26}
 Peebles P.J. E. and B. Ratra, The cosmological constant and dark energy, astro-ph/0207347. 
 \bibitem{27}
  Carroll, S.M. “The Cosmological Constant”, Living Rev. Relativity, 4, (2001),1, http://www.livingreviews.org/Articles/Volume4/2001-1carroll; Carroll, S.M., Press, W.H. and Turner, E.L. (1992) Ann. Rev. Astron. Astrophys. 30 499.
 \bibitem{28}
 Yin, Xinyou, J. A. N. Goudriaan, Egbert A. Lantinga, J. A. N. Vos, and Huub J. Spiertz. "A flexible sigmoid function of determinate growth." Annals of botany 91, no. 3 (2003): 361-371.
\bibitem{29} 
Danca, M.-F.: Continuous approximations of a class of piecewise continuous systems. Int. J. Bifurcation Chaos 25(11), 1550146 (2015). https://doi.org/10.1142/ S0218127415501461
\bibitem{30}
 Martinelli, Cristiano, Andrea Coraddu, and Andrea Cammarano. "Approximating piecewise nonlinearities in dynamic systems with sigmoid functions: advantages and limitations." Nonlinear Dynamics 111, no. 9 (2023): 8545-8569.
\bibitem{31}
 Riess, Adam G., Alexei V. Filippenko, Peter Challis, Alejandro Clocchiatti, Alan Diercks, Peter M. Garnavich, Ron L. Gilliland et al. "Observational evidence from supernovae for an accelerating universe and a cosmological constant." The astronomical journal 116, no. 3 (1998): 1009.
 \bibitem{32}
 Perlmutter, Saul, Goldhaber Aldering, Gerson Goldhaber, Richard A. Knop, Peter Nugent, Patricia G. Castro, Susana Deustua et al. "Measurements of $\omega$ and $\lambda$ from 42 high-redshift supernovae." The Astrophysical Journal 517, no. 2 (1999): 565.
\bibitem{33}
 Hinshaw, Gary, D. Larson, Eiichiro Komatsu, David N. Spergel, CLaa Bennett, Joanna Dunkley, M. R. Nolta et al. "Nine-year Wilkinson Microwave Anisotropy Probe (WMAP) observations: cosmological parameter results." The Astrophysical Journal Supplement Series 208, no. 2 (2013): 19.
 \bibitem{34}
 Aghanim, N. "Planck 2018 results. VI. Cosmological parameters." Astron. Astrophys 641 (2020): A6.
 \bibitem{35}
 Eisenstein, Daniel J., Idit Zehavi, David W. Hogg, Roman Scoccimarro, Michael R. Blanton, Robert C. Nichol, Ryan Scranton et al. "Detection of the baryon acoustic peak in the large-scale correlation function of SDSS luminous red galaxies." The Astrophysical Journal 633, no. 2 (2005): 560.
\bibitem{36}
 Copeland, Edmund J., Mohammad Sami, and Shinji Tsujikawa. "Dynamics of dark energy." International Journal of Modern Physics D 15, no. 11 (2006): 1753-1935.
\bibitem{37}
 Vakili, Babak, and Mohammad A. Gorji. "Thermostatistics with minimal length uncertainty relation." Journal of Statistical Mechanics: Theory and Experiment 2012, no. 10 (2012): P10013; Jochim, Selim, Markus Bartenstein, Alexander Altmeyer, Gerhard Hendl, Stefan Riedl, Cheng Chin, J. Hecker Denschlag, and Rudolf Grimm. "Bose-Einstein condensation of molecules." Science 302, no. 5653 (2003): 2101-2103.
\bibitem{38}
Nojiri, Shin'ichi, Sergei D. Odintsov, and Sachiko Ogushi. "Logarithmic corrections to the FRW brane cosmology from 5d Schwarzschild–de Sitter black hole." International Journal of Modern Physics A 18, no. 19 (2003): 3395-3416.
\bibitem{39} 
Lidsey, J.E., Nojiri, S., Odintsov, S.D., and Ogushi, S., “The AdS/CFT correspondence and logarithmic corrections to braneworld cosmology and the Cardy–Verlinde formula”, Phys. Lett. B, 544, 337–345, (2002). [hep-th/0207009]
\bibitem{40}
S. Das and R. K. Bhaduri, “Bose-Einstein condensate in cosmology,” arXiv:1808.10505 [gr-qc]; "Dark matter and dark energy from a Bose–Einstein condensate." Classical and Quantum Gravity 32, no. 10 (2015): 105003.
 \bibitem{41} 
 Koussour, M., N. S. Kavya, V. Venkatesha, and N. Myrzakulov. "Cosmic expansion beyond $\Lambda$CDM: investigating power-law and logarithmic corrections." The European Physical Journal Plus 139, no. 2 (2024): 1-13. 
\bibitem{42}
L. Randall and R. Sundrum, “A large mass hierarchy from a small extra dimension”, Phys. Rev. Lett. 83, 3370 (1999) [hep-ph/9905221], L. Randall and R. Sundrum, “An alternative to compactification”, Phys. Rev. Lett. 83, 4690 (1999) [hep-th/9906064]
\bibitem{43} 
Brax, Philippe, and Carsten van de Bruck. "Cosmology and brane worlds: a review." Classical and Quantum Gravity 20, no. 9 (2003): R201.
\bibitem{44}
 Maartens, Roy, and Kazuya Koyama. "Brane-world gravity." Living Reviews in Relativity 13 (2010): 1-124.
\bibitem{45}
 L. A. Gergely, E¨otv¨os branes, Phys. Rev. D 79 (2009) 086007 [arXiv: 0806.4006 [gr-qc]]. 
\bibitem{46}
 Gergely, László Árpád. "Friedmann branes with variable tension." Physical Review D—Particles, Fields, Gravitation, and Cosmology 78, no. 8 (2008): 084006 [arXiv: 0806.3857 [gr-qc]].
\bibitem{47}
Fukuyama, Takeshi, and Masahiro Morikawa. "The relativistic gross-pitaevskii equation and cosmological bose-einstein condensation: Quantum structure in the universe." Progress of theoretical physics 115, no. 6 (2006): 1047-1068; T. Fukuyama and M. Morikawa, Phys. Rev. D 80, 063520 (2009); Harko, T. "Cosmological dynamics of dark matter Bose-Einstein condensation." Physical Review D—Particles, Fields, Gravitation, and Cosmology 83, no. 12 (2011): 123515.
\bibitem{48}
Wong, K. C., K. S. Cheng, and T. Harko. "Inflation and late-time acceleration in braneworld cosmological models with varying brane tension." The European Physical Journal C 68 (2010): 241-253.
\bibitem{49}
Garcia-Aspeitia, Miguel A., A. Hernandez-Almada, Juan Magana, Mario H. Amante, V. Motta, and C. Martínez-Robles. "Brane with variable tension as a possible solution to the problem of the late cosmic acceleration." Physical Review D 97, no. 10 (2018): 101301.
\bibitem{50}
 Langlois, D., "Brane cosmology: an introduction." Prog. Theor. Phys. Suppl. 148, 181 (2003). [arXiv:hep-th/0209261]
\bibitem{51}
 Dabrowski, Mariusz P., Włodzimierz Godłowski, and Marek Szydłowski. "Brane universes tested against astronomical data." International Journal of Modern Physics D 13, no. 08 (2004): 1669-1702.
\bibitem{52} 
Binetruy et al., Nucl. Phys. B 565, 269 (2000), arXiv:hep-th/9905012.
\bibitem{53}
Planck Collaboration, A and A 641, A6 (2020).
\bibitem{54}
Garcia-Aspeitia et al., JCAP 07, 034 (2016), arXiv:1609.08220.


\end{thebibliography}



\end{document}
